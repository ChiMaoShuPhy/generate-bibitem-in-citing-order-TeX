\documentclass[english,preprint,aps,prd,showpacs,superscriptaddress,nofootinbib,tightenlines]{revtex4}
\bibliographystyle{unsrt}
\usepackage[T1]{fontenc}
\usepackage[latin9]{inputenc}
\setcounter{secnumdepth}{3}
\usepackage{float}
\usepackage{amsmath}
\usepackage{graphicx}
\usepackage{esint}
\usepackage{ulem}
\usepackage{amsfonts}
\usepackage{multirow}
\usepackage{mathrsfs}
\usepackage{graphicx}
\usepackage{amsmath}
\usepackage{amssymb}
\usepackage{bm}
\usepackage{bbm}
\usepackage{slashbox}
\usepackage{xcolor}
\usepackage{babel}
%%%%%%%%%%%%%%%%%%%%%%%%%%%%%%%%%%%%%%%%%
%Put your definitions here
%%%%%%%%%%%%%%%%%%%%%%%%%%%%%%%%%%%%%%%%%%
%Put your definitions here

\def\bl{\mbox{\scriptsize\boldmath $\ell$}}
\def\bll{\mbox{\boldmath $\ell$}}
\def\bfnabla{\mbox{\boldmath $\nabla$}}
\def\bfalpha{\mbox{\boldmath $\alpha$}}
\def\bfpsi{\mbox{\boldmath $\psi$}}
\def\bfeta{\mbox{\boldmath $\eta$}}
\def\bfchi{\mbox{\boldmath $\chi$}}
\def\bfgamma{\mbox{\boldmath $\gamma$}}
\def\bfSigma{\mbox{\boldmath $\Sigma$}}
\def\bfsigma{\mbox{\boldmath $\sigma$}}
\def\bflambda{\mbox{\boldmath $\lambda$}}
\def\bfrho{\mbox{\boldmath $\rho$}}
\def\bfPi{\mbox{\boldmath $\Pi$}}
\def\bfxi{\mbox{\boldmath $\xi$}}
\def\bfepsilon{\mbox{\boldmath $\epsilon$}}
\def\bfvarepsilon{\mbox{\boldmath $\varepsilon$}}

\makeatletter
%%%%%%%%%%%%%%%%%%%%%%%%%%%%%% Textclass specific LaTeX commands.
\@ifundefined{textcolor}{}
{%
 \definecolor{BLACK}{gray}{0}
 \definecolor{WHITE}{gray}{1}
 \definecolor{RED}{rgb}{1,0,0}
 \definecolor{GREEN}{rgb}{0,1,0}
 \definecolor{BLUE}{rgb}{0,0,1}
 \definecolor{CYAN}{cmyk}{1,0,0,0}
 \definecolor{MAGENTA}{cmyk}{0,1,0,0}
 \definecolor{YELLOW}{cmyk}{0,0,1,0}
}

\makeatother

\usepackage{babel}



\begin{document}

\title{Test File for bibgen.nb}

\author{Yu Jia\footnote{jiay@ihep.ac.cn}}
\affiliation{Institute of High Energy Physics and Theoretical Physics Center for
Science Facilities, Chinese Academy of
Sciences, Beijing 100049, China\vspace{0.2cm}}
\affiliation{Center
for High Energy Physics, Peking University, Beijing 100871,
China\vspace{0.2cm}}

\author{Xiaonu Xiong\footnote{xiaonu.xiong@pv.infn.it}}

\affiliation{Istituto Nazionale di Fisica Nucleare, Sezione di Pavia, Pavia, 27100,
Italy}


\date{\today}

\begin{abstract}
The recently-proposed quasi distributions point out a promising direction
for lattice QCD to investigate the light-cone correlators,
such as parton distribution functions (PDF) and distribution amplitudes (DA),
directly in the $x$-space.
Owing to its excessive simplicity,
the heavy quarkonium can serve as an ideal theoretical laboratory to ascertain
certain features of quasi-DA.
In the framework of non-relativistic QCD (NRQCD) factorization, we compute the order-$\alpha_s$ correction to
both light-cone distribution amplitudes (LCDA) and quasi-DA associated with the lowest-lying quarkonia,
with the transverse momentum UV cutoff interpreted as the renormalization scale.
We confirm analytically that the quasi-DA of a quarkonium does reduce to
the respective LCDA in the infinite-momentum limit.
We also observe that, provided that the momentum of a charmonium reaches about
2-3 times its mass, the quasi-DAs already converge to the LCDAs to a decent level.
These results might provide some useful guidance for the future
lattice study of the quasi distributions.
\end{abstract}

%%%%%%%%%%%%%%%%%%%%%%%%%%%%%%%%%%%%%%%%%%%%%%%%%%%%%%%%%%%%%%%%%%%%%%%%%%%%%%
\pacs{\it 12.38.Bx, 12.38.Gc, 14.40.Pq}

%11.10.Hi Renormalization group evolution of parameters
%12.38.Bx Perturbative calculations
%12.38.Cy Summation of perturbation theory
%12.38.-t Quantum chromodynamics
%12.39.St Factorization
%13.20.Gd Decays of J/¦×, ¦´, and other quarkonia
%13.60.Le Meson production
%13.60.Hb Total and inclusive cross sections (including deep-inelastic processes)
%13.87.Fh Fragmentation into hadrons
%14.40.Pq Heavy quarkonia
%%%%%%%%%%%%%%%%%%%%%%%%%%%%%%%%%%%%%%%%%%%%%%%%%%%%%%%%%%%%%%%%%%%%%%%%%%%%%%

\maketitle

\section{introduction}

The QCD factorization theorems~\cite{Collins:1989gx} imply that the
parton distribution functions (PDF)~\cite{Collins:1981uw} play the central role in accounting for virtually every
high-energy collision experiment.
In addition to PDF, there also exist other important types of light-cone correlators,
such as generalized parton distributions (GPD), transverse momentum dependent distributions
(TMDs), and light-cone distribution amplitudes (LCDA), all of which probe the internal structure of a hadron
in terms of fundamental quark-gluon degree of freedom.

These light-cone correlators are of nonperturbative nature, and are notoriously difficult to compute from
the first principle of QCD. The eminent obstacle for the lattice simulation originates from the fact that
they are defined in terms of the bilocal operators with light-like separation.
In the past,
lattice simulation has mainly focused on computing their
moments~\cite{Alexandrou:2015qia,Hagler:2007xi,Musch:2011er,Braun:2015lfa},
which are constructed out of the local operators.
Unfortunately, it becomes quickly impractical to go beyond a first few moments, since the more derivatives added,
the noisier the lattice simulation would become. To date,
our comprehensive knowledge about the nucleon PDF is gleaned exclusively through extracting from the
experimental data~\cite{Gao:2013xoa,Martin:2009iq,Ball:2014uwa}.

An exciting breakthrough has emerged recently.
A lattice calculation scheme directly in $x$-space was proposed by Ji in 2013~\cite{Ji:2013dva}.
In this approach, the task of computing the original light-cone correlators is transformed into computing
a new class of nonlocal matrix elements: the so-called {\it quasi} distributions.
These quasi distributions are defined as equal-time yet spatially-nonlocal correlation functions, thus amenable to
the lattice simulation.
In contrast to the light-cone quantities, the quasi distributions are generally frame-dependent.
But in the infinite momentum frame (IMF), the quasi distributions are expected to exactly recover the original
light-cone distributions. Ji has further envisaged that,
in analogy with the heavy quark effective theory (HQET), the quasi distribution method can be framed in an effective field theory context,
dubbed {\it Large Momentum Effective field Theory} (LaMET)~\cite{Ji:2014gla}.
The LaMET was first applied to proton spin structure,
which provides a means to extract the nucleon spin contents from the quasi distributions calculated on lattice~\cite{Ji:2013fga,Ji:2014lra}.

The utility of this new approach hinges crucially on the key that the quasi-distributions and
light-cone distribution share the exactly same infrared (IR) properties. It implies there exists a factorization theorem that connects these two quantities,
with perturbatively calculable matching coefficients.
Once the lattice has measured the quasi distributions, one can use this factorization formula to reconstruct
the desired light-cone quantities.

During the past two years, the one-loop matching factors have been computed for PDFs, GPDs for the non-singlet quark,
as well as pion DA~\cite{Xiong:2013bka,Ma:2014jla,Ji:2015qla}. The quasi TMD was also studied in \cite{Ji:2014hxa}.
Very recently, the two-loop renormalization of quasi-PDF has also been conducted~\cite{Ji:2015jwa}.
The factorization theorem for PDF has recently been proved to all orders in $\alpha_s$~\cite{Ma:2014jla}.
In addition, there recently have emerged some preliminary results from exploratory lattice simulations, extracting
the PDF from quasi PDF through the matching procedure outlined above~\cite{Lin:2014zya,Alexandrou:2015rja}.

To turn the quasi-distributions into a fruitful industry,
there remain many technical obstacles to overcome.
One outstanding challenge is to systematically implement the renormalization of such nonlocal operators on lattice.
Another difficulty stems from the technical limitation that, it is too expensive for the current lattice resources to
accommodate a fast-moving hadron on the lattice, since it requires exquisitely fine lattice spacing.
It is fair to say that, there is still a long way to go for the lattice simulation to be able to produce
phenomenologically competitive results.

For the lack of nonperturbative understanding of quasi distributions,
it is worth looking at their features from the perspective of phenomenological models.
For example, very recently the nucleon quasi-PDF has been investigated in a diquark model~\cite{Gamberg:2014zwa},
and the authors have examined how fast the nucleon quasi-PDF would approach the PDF with the increasing nucleon momentum.

Needless to say, it is also highly desirable to gain understanding about the gross features of the quasi distributions
from a model-independent angle. This consideration has motivated us
to study the distribution amplitudes (DA) of heavy quarkonia, chiefly because they offer
a unique, clean platform to scrutinize the quasi distributions. The key reason is that the DA of quarkonium
can be largely understood solely within perturbation theory.

The widely-separated scales ($m\gg mv,\,\Lambda_{\rm QCD}$) inherent to quarkonium
invites an effective-field-theory treatment.
In fact, the influential non-relativistic QCD (NRQCD) factorization approach~\cite{Bodwin:1994jh},
which fully exploits this scale hierarchy, nowadays has become an indispensable tool to tackle
quarkonium-related phenomena.

According to NRQCD factorization, the LCDA of a heavy quarkonium can be factorized as
the sum of the product of perturbatively-calculable, IR-finite coefficient functions and nonperturbative local NRQCD matrix elements~\cite{Ma:2006hc,Bell:2008er,Wang:2013ywc}.
At the lowest order in velocity expansion, up to a normalization factor,
the profile of the quarkonium LCDA is fully amenable to perturbation theory.

In this work, we generalize this knowledge and apply NRQCD factorization further to
the quasi-DA of heavy quarkonia, and calculate the respective coefficient functions to order $\alpha_s$.
To keep things as simple as possible, we concentrate on the lowest-lying $S$-wave quarkonia.
We have verified that, like the LCDAs of quarkonium, the quasi-DAs at order $\alpha_s$ are also IR-finite.
We are able to show analytically that, the quasi-DA exactly reduces to the LCDA in the infinity-momentum limit.
We also observe that, provided that the quarkonium is boosted to carry a momentum about 2-3 times its mass,
and with the renormalization scale chosen around the charmonium mass,
the respective quasi-DAs will converge to the LCDAs to a satisfactory degree.

We hope some of features about the quarkonium quasi-DAs may also apply to other hadrons.
Hopefully this knowledge will provide some useful guidance to the future lattice investigation
of similar quasi distributions.


%%%%%%%%%%%%%%%%%%%%%%%%%%%%%%%%%%%%%%%%%%%%%%%%%%%%%%%%%%%%%%%%%%%%%%%%%%%%%
\section{NRQCD factorization of Quarkonium Distribution Amplitudes}
\label{NRQCD:fac:LC:quasi}
%%%%%%%%%%%%%%%%%%%%%%%%%%%%%%%%%%%%%%%%%%%%%%%%%%%%%%%%%%%%%%%%%%%%%%%%%%%%%

In contrast to the light hadrons, heavy quarkonia are arguably among the simplest hadrons:
its constituent quark and antiquark are quite heavy, $m\gg \Lambda_{\rm QCD}$,
and move rather slowly ($v\ll 1$). These two essential features result in the
hierarchical structure of intrinsic energy scales of a quarkonium.
NRQCD factorization approach~\cite{Bodwin:1994jh} fully exploits this scale hierarchy,
and allows one to efficiently separate the relativistic/perturbative contributions from the long-distance/nonperturbative dynamics.
For most quarkonium-related phenomena, {\it i.e.} quarknoium production and decay processes,
this factorization approach has become an standard tool.

It is well known that the fragmentation functions for a parton transitioning into a light hadron
are genuinely nonperturbative
objects, and the only way to extract them is through experimental measurements~\cite{Agashe:2014kda}.
On the contrary, it was realized long ago that the heavy quarkonium fragmentation function can be
put in a factorized form~\cite{Braaten:1993mp,Braaten:1993rw}.
Concretely speaking, for a gluon-to-quarkonium fragmentation function, one has
%--------------------------
\begin{align}
D_{g\to H+X}(z,\mu) &= \sum_n  d_{g\to c\bar{c}[n]}(z) \langle 0| O_H[n] |0\rangle,
\end{align}
%---------------------------
where $z$ denotes the momentum fraction, and $n$ specifies the color/spin/orbital quantum number of the $c\bar{c}$ pair, and
$O_H[n]$ is the NRQCD four-fermion operators, which characterizes the transition probability from the
partonic state $c\bar{c}[n]$ to the quarkonium $H$ plus additional soft hadrons.
The key insight is that coefficient functions $d_{g\to c\bar{c}[n]}(z)$ are perturbatively calculable.

Analogous to the case of aforementioned fragmentation function,
one might naturally envisage that the DA of a quarkonium is also not a fully nonpertubative object,
and some sort of short-distance ($\sim 1/m$) effects should be disentangled owing to asymptotic freedom.
Indeed, such an analogy has already been pursued some time ago~\cite{Ma:2006hc,Bell:2008er}.
Schematically, one may express the quarkonium DA in the following factorized form:
\begin{align}
\Phi_{H}(x,\mu) \sim & \sum_{n} \langle H\left| \mathcal{O}_{[n]}\right|0\rangle
\phi_{H[n]}(x,\mu),
\label{eq:ReFact}
\end{align}
where the color-singlet NRQCD operators $\mathcal{O}_{[n]}$ are
organized according to the importance in the velocity expansion. Apart from the universal NRQCD matrix elements,
the key observation is that $\phi_{H[n]}\left(x,\mu\right)$ can now be interpreted as the short-disance coefficients.
Actually, for the hard exclusive quarkonium production, employing this factorized quarkonium LCDA turns out to
have considerable advantage compared with conventional NRQCD factorization approach~\cite{Jia:2008ep,Jia:2010fw}.

For simplicity, in this work we will only concentrate on the distribution amplitudes of $S$-wave quarkonia. Moreover,
we will only be interested in the lowest order in $v$ expansion.
Obviously, there is no any principal difficulty to incorporate the relativistic corrections,
or even extend to higher orbital quarkonium states.

\subsection{NRQCD factorization of quarkonium LCDA}

To be specific, let us assume the quarkonium $H$ to move along the positive $z$-axis,
{\it i.e.}, $P^{\mu}=\left(\sqrt{P_z^2+m^{2}},\boldsymbol{0}_{\perp}, P^{z} \right)$ with $P^z>0$.
For a general 4-vector $V^\mu$, it is convenient to introduce the light-cone plus (minus) components
 $V^{\pm}={1\over \sqrt{2}}(V^{0}\pm V^{z})$.

The leading-twist LCDAs of the pseudoscalar meson $P$, longitudinally (transversely) polarized vector meson $V^{\parallel,\perp}$,
are defined as
%---------------------------
\begin{subequations}
%---------------------------
\begin{align}
%---------------------------
\Phi_{P}\left(x,\mu\right)= & -if_{P} P^{+} \phi_{P}\left(x,\mu\right)\nonumber \\
%---------------------------
= & \int\frac{d\xi^{-}}{2\pi}\,e^{-i\left(x-\frac{1}{2}\right) P^{+}\xi^{-}}\left\langle P\left(P\right)\!\left|\!\bar{\psi}\left(\frac{\xi^{-}}{2}\right)\gamma^{+}\gamma^{5}\mathcal{W}
\psi\left(-\frac{\xi^{-}}{2}\right)\!\right|0\right\rangle ,\\
%---------------------------
\Phi_{V}^{\parallel}\left(x,\mu\right)= & -if_{V}^{\parallel}\varepsilon_{\parallel}^{*+} M_{V} \phi_{V}^{\parallel}\left(x,\mu\right)\nonumber \\
%---------------------------
= & \int\frac{d\xi^{-}}{2\pi}\,e^{-i\left(x-\frac{1}{2}\right) P^{+}\xi^{-}}\left\langle V\left(P,\varepsilon_\parallel\right)\!\left|\!\bar{\psi}\left(\frac{\xi^{-}}{2}\right)\gamma^{+}
\mathcal{W} \psi\left(-\frac{\xi^{-}}{2}\right)\!\right|0\right\rangle ,\\
%---------------------------
\Phi_{V}^\perp\left(x,\mu\right)= & -i f_{V}^{\perp} P^{+} \phi_{V}^\perp \left(x,\mu\right)
%---------------------------
\nonumber \\
%---------------------------
= & \int\frac{d\xi^{-}}{2\pi}\,e^{-i\left(x-\frac{1}{2}\right)P^{+}\xi^{-}}\left\langle V\left(P,\varepsilon_\perp \right)\!\left|\!\bar{\psi}\left(\frac{\xi^{-}}{2}\right)\gamma^{+} \bfgamma \cdot \bfvarepsilon_{\perp}
\mathcal{W} \psi\left(-\frac{\xi^{-}}{2}\right)\!\right|0\right\rangle,
%---------------------------
\end{align}
\label{eq:LCDA_Def}
\end{subequations}
%---------------------------
where $\varepsilon_{\parallel}^{\mu}$,
$\epsilon_{\perp}^{\mu}$ are the polarization vector for longitudinally
and transversely polarized vector meson, $\mu$ signifies the renormalization scale. $\mathcal{W}$ is the gauge
link along the light-cone ``minus'' direction:
%---------------------------
\begin{align}
%---------------------------
\mathcal{W}  &= {\mathcal P} \exp\left[-i g_s
\int^{\xi^{-}\over 2}_{-{\xi^{-}\over 2}}
d \eta^- A^+(\eta^-)\right].
%---------------------------
\end{align}
%---------------------------

The decay constants $f_{H}$ are defined as the vacuum-to-quarkonium
matrix elements mediated by various local QCD currents:
%---------------------------
\begin{subequations}
\begin{align}
\langle P (P) |\bar{\psi}\gamma^{+}\gamma^{5}\psi|0\rangle & \equiv  -i  f_{P} P^{+}= \int_{0}^{1}dx\,\Phi_{P}\left(x,\mu\right),\\
%---------------------------
\left\langle V(P,\varepsilon_{\parallel})\left|\bar{\psi} \gamma^{+}\psi \right|0\right\rangle & \equiv -i M_V
f_V^{\parallel} \varepsilon_{\parallel}^{*+} = \int_{0}^{1}dx\,\Phi_{V}^{\parallel} \left(x,\mu\right),\\
%---------------------------
\left\langle V(P,\varepsilon_{\perp})\left| \bar{\psi} \gamma^+ \bfgamma_\perp \psi \right|0\right\rangle & \equiv
-i f_{V}^\perp P^{+} \bfvarepsilon_{\perp}^{*}= \int_{0}^{1}dx\,\Phi_V^\perp(x,\mu).
\end{align}
\label{Def:decay:constant:LC}
\end{subequations}

The LCDA is clearly subject to the normalization condition:
%--------------------------
\begin{align}
\int^1_0 \! dx\,\phi_H(x)=1 \qquad{\rm for} \;\;\forall\;\; H.
\end{align}
%---------------------------

Thus far, everything is about the standard definition, valid for any pseudoscalar and vector mesons.
So what is special about the heavy quarkonium? As has been argued previously, the quarkonium DA defined above
still contains short-distance contribution, which ought to be identified and isolated.

If $H$ is a $S$-wave quarkonium state, the precise implication of NRQCD factorization of the LCDA is
%--------------------------------------------
\begin{subequations}
\begin{align}
%--------------------------------------------
\phi_H(x) &= \phi^{(0)}_H(x) + {C_F\alpha_s  \over \pi}\phi^{(1)}_H(x) +\cdots, \label{LCDA:pert:expansion} \\
%--------------------------------------------
f_{H} &= f_{H}^{(0)} \left(1+{C_F\alpha_s \over \pi}\,{\mathfrak f}_{H}^{(1)}+\cdots \right)+O(v^2), \label{Decay:constant:matching}
%--------------------------------------------
\end{align}
\label{Precise:meaning:NRQCD;fac:LCDA}
\end{subequations}
%--------------------------------------------
where $H= P, V_\parallel, V_\perp$. For the DAs of the hidden-flavor quakonia (the $c\bar{c}$ or $b\bar{b}$ family),
charge conjugation symmetry demands that they are symmetric under $x\leftrightarrow 1-x$.

The key message conveyed in (\ref{Precise:meaning:NRQCD;fac:LCDA}) is that the
$\phi_H(x)$ entailing all the hard ``collinear'' degree of freedom
(with typical virtuality of order $m^2$), thus can be computed in perturbation theory owing to asymptotic freedom.
The nonperturbative aspects of quarkonium are encoded
in the decay constant $f_H$.
Moreover, as indicated in (\ref{Decay:constant:matching}), one can match the QCD currents to the respective NRQCD
currents, by integrating out the hard quantum fluctuation. Consequently,
the genuinely nonperturbative binding dynamics is
encapsulated in the NRQCD matrix elements $f_{H}^{(0)}$.
For $H=\eta_c,\,J/\psi$, one has
%--------------------------------------------
\begin{subequations}
\begin{align}
f_{\eta_c}^{(0)} &= {1\over \sqrt{m_c}} \langle \eta_c| \psi^\dagger \chi|0\rangle  \approx
\sqrt{N_c\over 2\pi m_c} R_{\eta_c}(0), \\
f_{J/\psi}^{\parallel(0)} &= f_{J/\psi}^{\perp (0)}= {1\over \sqrt{m_c}} \langle J/\psi(\bfvarepsilon)| \psi^\dagger\bfsigma\cdot
\bfvarepsilon \chi|0\rangle \approx \sqrt{N_c\over 2\pi m_c} R_{J/\psi}(0),
\end{align}
\label{Def:NRQCD:vac:to:H:matrix:element}
\end{subequations}
%--------------------------------------------
where $\bfvarepsilon$ denotes the polarization three-vector in the $J/\psi$ rest frame,
and $N_c=3$ is the number of colors in QCD. Since NRQCD matrix elements are always defined in the quarkonium rest frame,
rotation invariance then implies that $f_{J/\psi}^{\parallel(0)} = f_{J/\psi}^{\perp(0)}$.
As implied in the last entity, these NRQCD matrix elements are often approximated by $R_H(0)$,
the radial Schr\"{o}dinger wave function at the origin for the $S$-wave charmonia,
which can be evaluated in the phenomenological quark potential models.



%-------------------------------

\begin{thebibliography}{99}


%\cite{Alexandrou:2015qia}
\bibitem{Alexandrou:2015qia}
  C.~Alexandrou, M.~Constantinou, S.~Dinter, V.~Drach, K.~Hadjiyiannakou, K.~Jansen, G.~Koutsou and A.~Vaquero,
  %``First moment of the flavour octet nucleon parton distribution function using lattice QCD,''
  JHEP {\bf 1506}, 068 (2015)
  [arXiv:1501.03734 [hep-lat]].
  %%CITATION = ARXIV:1501.03734;%%


%\cite{Collins:1989gx}
\bibitem{Collins:1989gx}
  J.~C.~Collins, D.~E.~Soper and G.~F.~Sterman,
  %``Factorization of Hard Processes in QCD,''
  Adv.\ Ser.\ Direct.\ High Energy Phys.\  {\bf 5}, 1 (1989)  [hep-ph/0409313].  %%CITATION = HEP-PH/0409313;%%  %586 citations counted in INSPIRE as of 12 Nov 2015

%\cite{Collins:1981uw}
\bibitem{Collins:1981uw}
  J.~C.~Collins and D.~E.~Soper,
  %``Parton Distribution and Decay Functions,''
  Nucl.\ Phys.\ B {\bf 194}, 445 (1982).
  %%CITATION = NUPHA,B194,445;%%
  %749 citations counted in INSPIRE as of 12 Nov 2015

%\cite{Hagler:2007xi}
\bibitem{Hagler:2007xi}
  P.~Hagler {\it et al.} [LHPC Collaboration],
  %``Nucleon Generalized Parton Distributions from Full Lattice QCD,''
  Phys.\ Rev.\ D {\bf 77}, 094502 (2008)
  [arXiv:0705.4295 [hep-lat]].
  %%CITATION = ARXIV:0705.4295;%%
  %233 citations counted in INSPIRE as of 12 Nov 2015

%\cite{Musch:2011er}
\bibitem{Musch:2011er}
  B.~U.~Musch, P.~Hagler, M.~Engelhardt, J.~W.~Negele and A.~Schafer,
  %``Sivers and Boer-Mulders observables from lattice QCD,''
  Phys.\ Rev.\ D {\bf 85}, 094510 (2012)
  [arXiv:1111.4249 [hep-lat]].
  %%CITATION = ARXIV:1111.4249;%%
  %52 citations counted in INSPIRE as of 12 Nov 2015

%\cite{Braun:2015lfa}
\bibitem{Braun:2015lfa}
  V.~M.~Braun, S.~Collins, M.~G?ckeler, P.~P¨¦rez-Rubio, A.~Sch?fer, R.~W.~Schiel and A.~Sternbeck,
  %``Pion Distribution Amplitude from Lattice QCD,''
  arXiv:1510.07429 [hep-lat].
  %%CITATION = ARXIV:1510.07429;%%

%\cite{Ma:2014jla}
\bibitem{Ma:2014jla}
Y.~-Q.~Ma and J.~-W.~Qiu,
%``Extracting Parton Distribution Functions from Lattice QCD Calculations,''
arXiv:1404.6860 [hep-ph].

%\cite{Gao:2013xoa}
\bibitem{Gao:2013xoa}
  J.~Gao {\it et al.},
  %``CT10 next-to-next-to-leading order global analysis of QCD,''
  Phys.\ Rev.\ D {\bf 89}, no. 3, 033009 (2014)
  [arXiv:1302.6246 [hep-ph]].
  %%CITATION = ARXIV:1302.6246;%%
  %257 citations counted in INSPIRE as of 12 Nov 2015


%\cite{Martin:2009iq}
\bibitem{Martin:2009iq}
  A.~D.~Martin, W.~J.~Stirling, R.~S.~Thorne and G.~Watt,
  %``Parton distributions for the LHC,''
  Eur.\ Phys.\ J.\ C {\bf 63}, 189 (2009)
  [arXiv:0901.0002 [hep-ph]].
  %%CITATION = ARXIV:0901.0002;%%
  %2955 citations counted in INSPIRE as of 12 Nov 2015

%\cite{Ball:2014uwa}
\bibitem{Ball:2014uwa}
  R.~D.~Ball {\it et al.} [NNPDF Collaboration],
  %``Parton distributions for the LHC Run II,''
  JHEP {\bf 1504}, 040 (2015)
  [arXiv:1410.8849 [hep-ph]].
  %%CITATION = ARXIV:1410.8849;%%
  %102 citations counted in INSPIRE as of 12 Nov 2015

\bibitem{Ji:2013dva}
X.~Ji,
%``Parton Physics on Euclidean Lattice,''
Phys.\ Rev.\ Lett.\  {\bf 110}, 262002 (2013) [arXiv:1305.1539 [hep-ph]].
%%CITATION = ARXIV:1305.1539;%%   %2 citations counted in INSPIRE as of

%\cite{Ji:2014gla}
\bibitem{Ji:2014gla}
  X.~Ji,
  %``Parton Physics from Large-Momentum Effective Field Theory,''
  Sci.\ China Phys.\ Mech.\ Astron.\  {\bf 57}, 1407 (2014)  [arXiv:1404.6680 [hep-ph]].
  %%CITATION = ARXIV:1404.6680;%%  %13 citations counted in INSPIRE as of 10 Nov 2015


%\cite{Ji:2013fga}
\bibitem{Ji:2013fga}
X.~Ji, J.~H.~Zhang and Y.~Zhao,
%``Physics of the Gluon-Helicity Contribution to Proton Spin,''
Phys.\ Rev.\ Lett.\  {\bf 111}, 112002 (2013) [arXiv:1304.6708 [hep-ph]].
%%CITATION = ARXIV:1304.6708;%%
%25 citations counted in INSPIRE as of 11 Nov 2015

%\cite{Ji:2014lra}
\bibitem{Ji:2014lra}
X.~Ji, J.~H.~Zhang and Y.~Zhao,
%``Justifying the Naive Partonic Sum Rule for Proton Spin,''
Phys.\ Lett.\ B {\bf 743}, 180 (2015)
[arXiv:1409.6329 [hep-ph]].
%%CITATION = ARXIV:1409.6329;%%
%11 citations counted in INSPIRE as of 11 Nov 2015

%\cite{Xiong:2013bka}
\bibitem{Xiong:2013bka}
X.~Xiong, X.~Ji, J.~H.~Zhang and Y.~Zhao,
%``One-loop matching for parton distributions: Nonsinglet case,''
Phys.\ Rev.\ D {\bf 90}, no. 1, 014051 (2014)   [arXiv:1310.7471 [hep-ph]].
%%CITATION = ARXIV:1310.7471;%%   %17 citations counted in INSPIRE as of 08 sept. 2015




%\cite{Lin:2014zya}
\bibitem{Lin:2014zya}
H.~-W.~Lin, J.~-W.~Chen, S.~D.~Cohen and X.~Ji,
%``Flavor Structure of the Nucleon Sea from Lattice QCD,''
arXiv:1402.1462 [hep-ph].
%%CITATION = ARXIV:1402.1462;%%   %2 citations counted in INSPIRE as of 07 May 2014

%\cite{Ji:2015qla}
\bibitem{Ji:2015qla}
X.~Ji, A.~Schäfer, X.~Xiong and J.~H.~Zhang,
%``One-Loop Matching for Generalized Parton Distributions,''
Phys.\ Rev.\ D {\bf 92}, no. 1, 014039 (2015)   [arXiv:1506.00248 [hep-ph]].
%%CITATION = ARXIV:1506.00248;%%

%\cite{Ji:2014hxa}
\bibitem{Ji:2014hxa}
X.~Ji, P.~Sun, X.~Xiong and F.~Yuan,
%``Soft Factor Subtraction and Transverse Momentum Dependent Parton Distributions on Lattice,''
Phys.\ Rev.\ D {\bf 91}, 074009 (2015) [arXiv:1405.7640 [hep-ph]].
%%CITATION = ARXIV:1405.7640;%%
%13 citations counted in INSPIRE as of 09 Nov 2015

%\cite{Ji:2015jwa}
\bibitem{Ji:2015jwa}
X.~Ji and J.~H.~Zhang,
%``Renormalization of quasiparton distribution,''
Phys.\ Rev.\ D {\bf 92}, 034006 (2015)   [arXiv:1505.07699 [hep-ph]].
%%CITATION = ARXIV:1505.07699;%%   %2 citations counted in INSPIRE as of 06 Nov 2015




%\cite{Gamberg:2014zwa}
\bibitem{Gamberg:2014zwa}
L.~Gamberg, Z.~B.~Kang, I.~Vitev and H.~Xing,
%``Quasi-parton distribution functions: a study in the diquark spectator model,''
Phys.\ Lett.\ B {\bf 743}, 112 (2015)   [arXiv:1412.3401 [hep-ph]].
%%CITATION = ARXIV:1412.3401;%%   %1 citations counted in INSPIRE as of 08 sept. 2015

%\cite{Alexandrou:2015rja}
\bibitem{Alexandrou:2015rja}
C.~Alexandrou, K.~Cichy, V.~Drach, E.~Garcia-Ramos, K.~Hadjiyiannakou, K.~Jansen, F.~Steffens and C.~Wiese,
%``Lattice calculation of parton distributions,''
Phys.\ Rev.\ D {\bf 92}, no. 1, 014502 (2015)   [arXiv:1504.07455 [hep-lat]].
%%CITATION = ARXIV:1504.07455;%%   %1 citations counted in INSPIRE as of 08 sept. 2015



%\cite{Bodwin:1994jh}
\bibitem{Bodwin:1994jh}
  G.~T.~Bodwin, E.~Braaten and G.~P.~Lepage,
  %``Rigorous QCD analysis of inclusive annihilation and production of heavy quarkonium,''
  Phys.\ Rev.\ D {\bf 51}, 1125 (1995)  [Phys.\ Rev.\ D {\bf 55}, 5853 (1997)]  [hep-ph/9407339].
  %%CITATION = HEP-PH/9407339;%%  %1807 citations counted in INSPIRE as of 10 Nov 2015


%\cite{Braaten:1993rw}
\bibitem{Braaten:1993rw}
  E.~Braaten and T.~C.~Yuan,
  %``Gluon fragmentation into heavy quarkonium,''
  Phys.\ Rev.\ Lett.\  {\bf 71}, 1673 (1993)  [hep-ph/9303205].
  %%CITATION = HEP-PH/9303205;%%  %256 citations counted in INSPIRE as of 10 Nov 2015


%\cite{Ma:2006hc}
\bibitem{Ma:2006hc}
  J.~P.~Ma and Z.~G.~Si,
  %``NRQCD Factorization for Twist-2 Light-Cone Wave-Functions of Charmonia,''
  Phys.\ Lett.\ B {\bf 647}, 419 (2007)  [hep-ph/0608221].
  %%CITATION = HEP-PH/0608221;%%  %33 citations counted in INSPIRE as of 10 Nov 2015

%\cite{Bell:2008er}
\bibitem{Bell:2008er}
G.~Bell and T.~Feldmann,
%``Modelling light-cone distribution amplitudes from non-relativistic bound states,''
JHEP {\bf 0804}, 061 (2008)   [arXiv:0802.2221 [hep-ph]].
%%CITATION = ARXIV:0802.2221;%%   %47 citations counted in INSPIRE as of 14 Oct 2015

%\cite{Wang:2013ywc}
\bibitem{Wang:2013ywc}
X.~P.~Wang and D.~Yang,
%``The leading twist light-cone distribution amplitudes for the S-wave and P-wave quarkonia and their applications in single quarkonium exclusive productions,''
JHEP {\bf 1406}, 121 (2014)   [arXiv:1401.0122 [hep-ph]].
%%CITATION = ARXIV:1401.0122;%%   %5 citations counted in INSPIRE as of 05 sept. 2015

%\cite{Agashe:2014kda}
\bibitem{Agashe:2014kda}
  K.~A.~Olive {\it et al.} [Particle Data Group Collaboration],
  %``Review of Particle Physics,''
  Chin.\ Phys.\ C {\bf 38}, 090001 (2014).
  %%CITATION = CHPHD,C38,090001;%%
  %2276 citations counted in INSPIRE as of 12 Nov 2015

%\cite{Braguta:2007fh}
%\bibitem{Braguta:2007fh}
%V.~V.~Braguta,   %``The study of leading twist light cone wave functions of J/psi meson,''
%Phys.\ Rev.\ D {\bf 75}, 094016 (2007)   [hep-ph/0701234 [HEP-PH]].
%%CITATION = HEP-PH/0701234;%%   %32 citations counted in INSPIRE as of 05 sept. 2015



%\cite{Braaten:1993mp}
\bibitem{Braaten:1993mp}
  E.~Braaten, K.~m.~Cheung and T.~C.~Yuan,
  %``Z0 decay into charmonium via charm quark fragmentation,''
  Phys.\ Rev.\ D {\bf 48}, 4230 (1993)  [hep-ph/9302307].
  %%CITATION = HEP-PH/9302307;%%  %205 citations counted in INSPIRE as of 10 Nov 2015



 %\cite{Jia:2010fw}
\bibitem{Jia:2010fw}
  Y.~Jia, J.~X.~Wang and D.~Yang,
  %``Bridging light-cone and NRQCD approaches: asymptotic behavior of $B_c$ electromagnetic form factor,''
  JHEP {\bf 1110}, 105 (2011)  [arXiv:1012.6007 [hep-ph]].  %%CITATION = ARXIV:1012.6007;%%  %12 citations counted in INSPIRE as of 12 Nov 2015

%\cite{Jia:2008ep}
\bibitem{Jia:2008ep}
  Y.~Jia and D.~Yang,
  %``Refactorizing NRQCD short-distance coefficients in exclusive quarkonium production,''
  Nucl.\ Phys.\ B {\bf 814}, 217 (2009)  [arXiv:0812.1965 [hep-ph]].  %%CITATION = ARXIV:0812.1965;%%  %16 citations counted in INSPIRE as of 12 Nov 2015


%\cite{Braaten:1996jt}
\bibitem{Braaten:1996jt}
E.~Braaten and Y.~Q.~Chen,
%``Helicity decomposition for inclusive J / psi production,''
Phys.\ Rev.\ D {\bf 54}, 3216 (1996)   [hep-ph/9604237].
%%CITATION = HEP-PH/9604237;%%   %110 citations counted in INSPIRE as of 28 Oct 2015


%\cite{Beneke:1997zp}
\bibitem{Beneke:1997zp}
  M.~Beneke and V.~A.~Smirnov,
  %``Asymptotic expansion of Feynman integrals near threshold,''
  Nucl.\ Phys.\ B {\bf 522}, 321 (1998)  [hep-ph/9711391].
  %%CITATION = HEP-PH/9711391;%%  %381 citations counted in INSPIRE as of 10 Nov 2015

%\cite{Braguta:2006wr}
\bibitem{Braguta:2006wr}
  V.~V.~Braguta, A.~K.~Likhoded and A.~V.~Luchinsky,
  %``The Study of leading twist light cone wave function of eta(c) meson,''
  Phys.\ Lett.\ B {\bf 646}, 80 (2007)
  [hep-ph/0611021].
  %%CITATION = HEP-PH/0611021;%%
  %42 citations counted in INSPIRE as of 12 Nov 2015


%\cite{Czarnecki:1997vz}
\bibitem{Czarnecki:1997vz}
  A.~Czarnecki and K.~Melnikov,
  %``Two loop QCD corrections to the heavy quark pair production cross-section in e+ e- annihilation near the threshold,''
  Phys.\ Rev.\ Lett.\  {\bf 80} (1998) 2531
  [hep-ph/9712222].
  %%CITATION = HEP-PH/9712222;%%
  %169 citations counted in INSPIRE as of 11 Apr 2015


%\cite{Beneke:1997jm}
\bibitem{Beneke:1997jm}
  M.~Beneke, A.~Signer and V.~A.~Smirnov,
  %``Two loop correction to the leptonic decay of quarkonium,''
  Phys.\ Rev.\ Lett.\  {\bf 80} (1998) 2535
  [hep-ph/9712302].
  %%CITATION = HEP-PH/9712302;%%
  %155 citations counted in INSPIRE as of 11 Apr 2015


%\cite{Czarnecki:2001zc}
\bibitem{Czarnecki:2001zc}
  A.~Czarnecki and K.~Melnikov,
  %``Charmonium decays: J / psi ---> e+ e- and eta(c) ---> gamma gamma,''
  Phys.\ Lett.\ B {\bf 519} (2001) 212
  [hep-ph/0109054].
  %%CITATION = HEP-PH/0109054;%%
  %31 citations counted in INSPIRE as of 11 Apr 2015

%\cite{Brodsky:1989pv}
\bibitem{Brodsky:1989pv}
  S.~J.~Brodsky and G.~P.~Lepage,
  %``Exclusive Processes in Quantum Chromodynamics,''
  Adv.\ Ser.\ Direct.\ High Energy Phys.\  {\bf 5}, 93 (1989).
  %%CITATION = 00319,5,93;%%
  %68 citations counted in INSPIRE as of 12 Nov 2015

%\cite{Lepage:1979zb}
\bibitem{Lepage:1979zb}
  G.~P.~Lepage and S.~J.~Brodsky,
  %``Exclusive Processes in Quantum Chromodynamics: Evolution Equations for Hadronic Wave Functions and the Form-Factors of Mesons,''
  Phys.\ Lett.\ B {\bf 87}, 359 (1979).
  %%CITATION = PHLTA,B87,359;%%
  %1113 citations counted in INSPIRE as of 12 Nov 2015

%\cite{Efremov:1979qk}
\bibitem{Efremov:1979qk}
  A.~V.~Efremov and A.~V.~Radyushkin,
  %``Factorization and Asymptotical Behavior of Pion Form-Factor in QCD,''
  Phys.\ Lett.\ B {\bf 94}, 245 (1980).
  %%CITATION = PHLTA,B94,245;%%
  %927 citations counted in INSPIRE as of 12 Nov 2015

 %\cite{Ma:2013yla}
\bibitem{Ma:2013yla}
  Y.~Q.~Ma, J.~W.~Qiu and H.~Zhang,
  %``Heavy quarkonium fragmentation functions from a heavy quark pair. I. $S$ wave,''
  Phys.\ Rev.\ D {\bf 89}, no. 9, 094029 (2014)
  [arXiv:1311.7078 [hep-ph]].
  %%CITATION = ARXIV:1311.7078;%%
  %25 citations counted in INSPIRE as of 12 Nov 2015

%\cite{Lepage:1980fj}
\bibitem{Lepage:1980fj}
  G.~P.~Lepage and S.~J.~Brodsky,
  %``Exclusive Processes in Perturbative Quantum Chromodynamics,''
  Phys.\ Rev.\ D {\bf 22}, 2157 (1980).
  %%CITATION = PHRVA,D22,2157;%%
  %2976 citations counted in INSPIRE as of 12 Nov 2015


%\bibitem{Braguta:2007fh}
%  V.~V.~Braguta,
%  %``The study of leading twist light cone wave functions of J/psi meson,''
%  Phys.\ Rev.\ D {\bf 75}, 094016 (2007)
%  [hep-ph/0701234 [HEP-PH]].
%  %%CITATION = HEP-PH/0701234;%%
%  %32 citations counted in INSPIRE as of 12 Nov 2015




\end{thebibliography}
\end{document}


